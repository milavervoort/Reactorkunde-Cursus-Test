\documentclass{ximera}

\title{Theory}
\author{Mila Vervoort}
\license{CC: 0}         % replace with an appropriate license, or set it in xmPreamble

\begin{document}
\begin{abstract}
    A simple theory part
\end{abstract}
\maketitle
\label{xim:theory}

In hoofdstuk 3 'Ontwerp van homogene reaktoren onder isotherme werkingsvoorwaarden' werd gezien hoe door het opstellen van de stofbalans de nodige ontwerpvergelijkingen bekomen kunnen worden. 

Een stofbalans is van de volgende vorm: 
\[
\text{Accumulation} = \text{In} - \text{Uit} + \text{Vorming} - \text{Verbruik}
\]
In formulevorm:
\[
\frac{dN_A}{dt} = F_{A,in} - F_{A,out} + R_A V
\]
waar:
\begin{itemize}
\item $N_A$ = aantal mol van component A in de reactor
\item $F_{A,in}$ = molstroom van A naar binnen
\item $F_{A,out}$ = molstroom van A naar buiten
\item $R_A$ = reactiesnelheid van A (mol/(m$^3$s))
\item $V$ = reactorvolume
\end{itemize}


\section*{1. Batch Reactor}

In een batch reactor is er geen in- of uitstroom (\(F_{A,in} = F_{A,out} = 0\)):

\[
\frac{dN_A}{dt} = R_A V
\]

Aanname dat het volume V constant blijft:

\[
\frac{dC_A}{dt} = R_A
\]


Om de tijd \(t\) te berekenen die nodig is om van een initiële concentratie \(C_{A0}\) naar \(C_A\) te gaan, integreren we:

\[
\frac{dC_A}{dt} = R_A(C_A)
\]

\[
\int_{t=0}^{t} dt = \int_{C_{A0}}^{C_A} \frac{dC_A}{R_A(C_A)}
\]

\[
t = \int_{C_A}^{C_{A0}} \frac{dC_A}{-R_A(C_A)}
\]

\textbf{Voorbeeld eerste orde reactie:} \(R_A = -k C_A\)

\[
\frac{dC_A}{dt} = -k C_A \quad \Rightarrow \quad \int_{C_{A0}}^{C_A} \frac{dC_A}{-k C_A} = \int_0^t dt
\]

\[
t = \frac{1}{k} \ln \frac{C_{A0}}{C_A}
\]

\section*{2. CSTR (Continuous Stirred Tank Reactor)}

In een CSTR met constante volumestroom \(Q\) geldt bij stationaire toestand (\(\frac{dN_A}{dt} = 0\)):

\[
F_{A,in} - F_{A,out} + R_A V = 0
\]

Invullen van \(F_A = C_A Q\) geeft:

\[
Q C_{A,in} - Q C_{A,out} + R_A V = 0
\]

of herschikt:

\[
C_{A,out} = C_{A,in} + \frac{R_A V}{Q}
\]

\textbf{Uitleg:} In een goed gemengde CSTR is de concentratie overal uniform. De uitstroomconcentratie wordt bepaald door de balans tussen toevoer en reactie.

\section*{3. PFR (Plug Flow Reactor)}

In een PFR verandert de concentratie langs de reactorlengte \(x\). De stofbalans over een klein segment \(dx\) is:

\[
F_A(x) - F_A(x+dx) + R_A dV = 0
\]

Bij een constante volumestroom \(Q\) (bij vloeistoffen):

\[
Q \frac{dC_A}{dx} = R_A
\]

Met \(dV = A dx\) (doorsnede \(A\)):

\[
\frac{dC_A}{dx} = \frac{R_A}{Q}
\]

\textbf{Uitleg:} In een PFR stroomt het reactiemengsel plug-gewijs; er is geen menging in de lengterichting. De concentratie verandert continu langs de reactor.

\section*{Samenvatting}

\begin{tabular}{|l|c|c|}
    \hline
    \textbf{Reactor Type} & \textbf{Stofbalans} & \textbf{Kernidee}\\
    \hline
    Batch & $\frac{dC_A}{dt} = R_A$ & Geen in-/uitstroom, alleen reactie\\
    CSTR & $Q C_{A,in} - Q C_{A,out} + R_A V = 0$ & Goed gemengd, stationaire toestand\\
    PFR & $\frac{dC_A}{dx} = \frac{R_A}{Q}$ & Concentratie verandert langs de lengte, plug-flow\\
    \hline
\end{tabular}


\end{document}